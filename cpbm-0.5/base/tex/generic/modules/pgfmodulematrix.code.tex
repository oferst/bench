% Copyright 2006 by Till Tantau
%
% This file may be distributed and/or modified
%
% 1. under the LaTeX Project Public License and/or
% 2. under the GNU Public License.
%
% See the file doc/generic/pgf/licenses/LICENSE for more details.

\ProvidesFileRCS $Header: /cvsroot/pgf/pgf/generic/pgf/modules/pgfmodulematrix.code.tex,v 1.8 2013/09/17 17:34:00 tantau Exp $

\usepgfmodule{shapes}

\newif\ifpgfmatrix
\newif\ifpgf@matrix@correct@call

\newif\ifpgf@matrix@fixed

\newcount\pgfmatrixcurrentrow

\newcount\pgfmatrixcurrentcolumn
\newcount\pgf@matrix@numberofcolumns

\let\pgf@matrix@cell=\pgfnodeparttextbox % alias
\let\pgf@matrix@box=\pgf@matrix@cell % alias


% Sets the default row or column sep
%
% #1 = default row or column sep list
%
% Description:
%
% Sets the macro \pgfmatrixrowsep or \pgfmatrixcolumnsep to #1.
%
% Example:
%
% \pgfsetmatrixrowsep{5pt}

\def\pgfsetmatrixrowsep#1{\def\pgfmatrixrowsep{#1}}
\def\pgfsetmatrixcolumnsep#1{\def\pgfmatrixcolumnsep{#1}}

\def\pgfmatrixrowsep{0pt}
\def\pgfmatrixcolumnsep{0pt}



% The pgfmatrix command creates a node that contains a matrix. Like
% the \pgfnode command, it is to be used inside a pgfpicture
% environment. 
%
% #1 = shape type
% #2 = anchor
% #3 = name for later reference
% #4 = path usage command
% #5 = inner shift
% #6 = startup code
% #7 = the matrix code (may contain verbatim stuff)
%
% Description:
%
% This command creates a node that contains a matrix. For 
% each cell of the matrix, drawing commands may be given. These
% commands will create a cell picture and a bounding box is computed
% for each cell picture. The cell picture is then placed inside the
% cell. The position inside the cell is computed according to the
% following rules: 
%
% 1. In each column, the origins of the cell pictures are all on a
%    vertical line.
% 2. In each row, the origins of the cell pictures are all on a
%    horizontal line.
% 3. All cells in the same row have the same height and depth.
% 4. All cells in the same column have the same width.
% 5. The size of cells are chosen minimal such that the boxes fit into
%    the cells and still meet the above requirement.
%
% Next, the whole resulting matrix, which is given as #7, is placed
% inside a node whose name is given as #3, whose shape is #1, and
% whose anchor is #2. This means that the resulting node is shifted
% such that the given anchor lies at the origin. Then, as for a normal
% node, the #4 path usage command is used for the node. 
%
% The parameter #5 is an additional negative shift for the node. Normally, such
% a shift could be given beforehand (that is, the shift could be
% preapplied to the current transformation matrix). However, when #5
% is evaluated, you can refer to temporary positions of nodes inside
% the matrix. In detail, the following happens: When the matrix has
% been typeset, all nodes in the matrix temporarily get assigned their
% positions in the matrix box. The origin of this coordinate system is
% at the left baseline end of the matrix box, which corresponds to the
% text anchor. The position #5 is then interpreted inside this
% coordinate system and then used for shifting. 
%
% The idea behind using the parameter #5 is the following: If you
% specify "text" as anchor #2 and specify \pgfpointanchor{inner
% node}{some anchor} for parameter #5, where inner node is a node that
% is created in the matrix, then the whole matrix will be shifted such
% that inner node.some anchor lies at the origin of the whole picture.
%
% At the beginning and at the end of each cell, the macros
% \pgfmatrixbegincode and \pgfmatrixendcode are called. Inside these
% macros you can use \pgfmatrixcurrentcolumn and \pgfmatrixcurrentrow
% to access the current row and column.
%
% Before each column except for the first, an extra space given by the
% macro \pgfmatrixcolumnsep is added. Additionally, an extra space
% given by the optional argument of the \pgfmatrixnextcell commands in
% the first line are added, which may vary between columns. Between
% rows, an extra space given by the macro \pgfmatrixrowsep is
% added. You can also change this value for each row by using a \gdef.
%
%
% Lines are ended using the command \pgfmatrixendrow. The command \\
% is \let to be the same, but \\ is sometimes overridden by, say,
% minipage, so it's good to have \pgfmatrixendrow as a "fallback". 
%
% Inside the matrix, \\ must be used to end every line ***including
% the last***. You can pass an optional parameter to \\ as in \\[2pt]
% to give an additional space after this particular line.
%
% The usage of \\ at the end of the matrix is imporant since this
% command is used to find the end of the matrix. What happens is that
% \\ followed by a closing brace ends the matrix. 
%
% Inside the matrix, you must use \pgfmatrixnextcell to separate
% columns. The reason is that two columns are used internally for
% every single column shown. The macro \pgfmatrixnextcell is yields
% (more or less) two && (and some further trickery).
%
% The parameter #6 is executed at the beginning of the matrix
% group. It is inside the protecting group of the matrix, but outside
% the matrix itself. It can be used for different purposes: 1) It can be
% used to simplify the next cell macro. 2) It can be used to issue an
% \aftergroup command. 3) It can be used to setup local settings. For
% example, by setting #6 to \let\&=\pgfmatrixnextcell you can use \&
% instead of \pgfmatrixnextcell. You can also set the catcode of & to
% active.
%
%
% Example:
%
% \begin{tikzpicture}
%
%   \draw (0,0) circle (1mm);  % b.north will lie on the center of this circle
%
%   \pgfmatrix{rectangle}{text}{mymatrix}{\pgfusepath{}}{\pgfpointanchor{b}{north}}{\let\&=\pgfmatrixnextcell}
%   {
%     \draw (0,0) circle (1cm); \& \node (a) [draw] {Hallo}; \\
%     \node (b) [draw] {Welt};  \& \draw (0,0) circle (5mm); \\
%   }
%
%   \draw [->] (a) -- (b);
%
% \end{tikzpicture}

\def\pgfmatrix#1#2#3#4#5#6{%
  \ifpgfmatrix%
    \pgferror{You cannot nest pgfmatrix environments, yet}
  \fi%
  \bgroup%  
    \everycr{}%
    #6%
    \def\pgf@matrix@par@shape{#1}%
    \def\pgf@matrix@par@anchor{#2}%
    \def\pgf@matrix@par@name{#3}%
    \def\pgf@matrix@par@usage{#4}%
    \def\pgf@matrix@par@shift{#5}%
    % the matrix must have a name:
    \ifx\pgf@matrix@par@name\pgfutil@empty%
      \def\pgf@matrix@par@name{pgf@matrix@internal}%
    \fi%
    \pgfmatrixcurrentrow=0\relax%
    \pgfmatrixcurrentcolumn=0\relax%
    \pgf@matrix@numberofcolumns=0\relax%
    \let\\=\pgfmatrixendrow%
    \tabskip=0pt%
    \offinterlineskip%
    \pgfmatrixtrue%
    \let\pgf@matrix@node@list=\pgfutil@empty%
    \expandafter\let\csname pgf@matrix@column@finish@1\endcsname=\pgfutil@empty%
    \let\pgf@matrix@row@finish=\pgfutil@empty%
    % 
    % Get rid of bgroup
    %
    \afterassignment\pgf@matrix@cont\let\pgf@next=%
}
\def\pgf@matrix@cont{%  
    \setbox\pgf@matrix@box=\hbox\bgroup\vbox\bgroup%
    \halign\bgroup%
    \pgf@matrix@init@row%
    \pgf@matrix@step@column%
    {%
      \pgf@matrix@startcell%
      ##%
      \pgf@matrix@endcell%
    }%
    &%
    ##\pgf@matrix@padding&&%
    \pgf@matrix@step@column%
    {%
      \pgf@matrix@startcell%
      ##%
      \pgf@matrix@endcell%
    }&%
    ##\pgf@matrix@padding%
    \cr%
}

\def\pgf@end@matrix{%
      \egroup\egroup%
    \egroup%
    %
    % If there are named nodes, we must adjust their coordinates.
    %
    \ifx\pgf@matrix@node@list\pgfutil@empty%
    \else%
      % Ok, this means we got a lot of work to do...
      \pgf@matrix@compute@origin%
      \pgf@matrix@shift@nodes@initial%
    \fi%
    %
    % Compute additional shift
    %
    \pgftransformresetnontranslations%
    \pgftransformshift{\pgf@process{\pgftransformreset\pgf@matrix@par@shift\pgf@x=-\pgf@x\pgf@y=-\pgf@y}}%
    %
    % Create node
    %
    \pgfmultipartnode{\pgf@matrix@par@shape}{\pgf@matrix@par@anchor}{\pgf@matrix@par@name}{\pgf@matrix@par@usage}%
    %
    % If there are named nodes, we must adjust their coordinates, again.
    %
    \ifx\pgf@matrix@node@list\pgfutil@empty%
    \else%
      {%
        \pgftransformreset%
        \pgf@process{\pgfpointanchor{\pgf@matrix@par@name}{text}}%
        \xdef\pgf@matrix@offset{\noexpand\pgfqpoint{\the\pgf@x}{\the\pgf@y}}%
        \pgf@matrix@shift@nodes@secondary{\pgf@matrix@offset}%
      }%
    \fi%
}




% Ending a cell
%
% #1 = (optional) spacing between previous and next column. This
%      parameter only has an effect in the first row.
%
% Description:
%
% This command ends a cell of a matrix.
%
% Example:
%
% \pgfmatrix{rectangle}{text}{mymatrix}{\pgfusepath{}}{\pgfpointanchor{b}{north}}{}
% {
%   \draw (0,0) circle (1cm); \pgfmatrixnextcell[1cm] \node (a) [draw] {Hallo}; \\
%   \node (b) [draw] {Welt};  \pgfmatrixnextcell \draw (0,0) circle (5mm); \\
% }

\def\pgfmatrixnextcell{%
  % if the cell contains nothing, the following \let will be at the
  % beginning (macro expansion has stopped here since neither \omit
  % nor \span was found)
  \let\pgf@matrix@signal@cell@end=\pgf@matrix@signal@cell@end%
  \pgfutil@ifnextchar[{\pgf@matrix@cell@cont}{\pgf@matrix@cell@cont[0pt]}}%}

\def\pgf@matrix@cell@cont[#1]{%
  \xdef\pgf@matrix@last@nextcell@options{#1}%
  &\pgf@matrix@correct@calltrue&%
}%


\def\pgf@matrix@fixed@text{between origins}
\def\pgf@matrix@natural@text{between borders}

\def\pgf@matrix@addtolength#1#2{\edef\pgf@temp{#2}\expandafter\pgf@matrix@addtolength@\expandafter#1#2,\pgf@stop}
\def\pgf@matrix@addtolength@#1#2,#3\pgf@stop{%
  \def\pgf@temp{#2}%
  \ifx\pgf@temp\pgfutil@empty%
  \else%
    \ifx\pgf@temp\pgf@matrix@fixed@text%
      \global\pgf@matrix@fixedtrue%
    \else%
      \ifx\pgf@temp\pgf@matrix@natural@text%
        \global\pgf@matrix@fixedfalse%
      \else%
        \pgfmathaddtolength{#1}{#2}%
      \fi%
    \fi%
  \fi%
  \def\pgf@temp{#3}%
  \ifx\pgf@temp\pgfutil@empty%
  \else%
    \pgf@matrix@addtolength@#1#3\pgf@stop%
  \fi%
}



% Init a line

\def\pgf@matrix@init@row{%
  \relax%
  \global\advance\pgfmatrixcurrentrow by 1\relax%
  \global\pgfmatrixcurrentcolumn=0\relax%
  \expandafter\gdef\csname pgf@matrix@maxy\the\pgfmatrixcurrentrow\endcsname{0pt}%
  \expandafter\gdef\csname pgf@matrix@miny\the\pgfmatrixcurrentrow\endcsname{0pt}%
}



% Step the row column number

\def\pgf@matrix@step@column{%
  \relax%
  \global\advance\pgfmatrixcurrentcolumn by1\relax%
}



% Start a cell

\def\pgf@matrix@startcell{%
  %
  % Step 1: Init the list of nodes for this cell
  %
  \let\pgf@nodecallback=\pgf@matrix@nodecallback%
  %
  % Step 2: Setup the bounding box
  %
  \pgfinterruptboundingbox%
  %
  % Step 3: Reset the transformation matrix
  %
  \pgftransformreset%
  %
  % Step 4: Collect everything in a cell box
  %
  \setbox\pgf@matrix@cell=\hbox\bgroup\bgroup%
	% make sure that cell pictures are not affected if matrizes have
	% 'overlay' option on:
	\pgf@relevantforpicturesizetrue
    \pgfsys@beginpicture%
    \normalbaselines%
    % Find out whether the cell is empty:
    \pgfutil@ifnextchar\let%
    {% ok, candidate, check following symbol
      \afterassignment\pgf@matrix@empty@check\let\pgf@matrix@temp=% get rid of \let
    }%
    {% no, not empty
      \pgf@matrix@empty@cell@false%
      \pgfmatrixbegincode%
    }%
}

\def\pgf@matrix@empty@check{\futurelet\pgf@matrix@check@token\pgf@matrix@empty@check@cont}
\def\pgf@matrix@empty@check@cont{%
  \ifx\pgf@matrix@check@token\pgf@matrix@signal@cell@end%
    \pgf@matrix@empty@cell@true%
    \expandafter%
    \pgfmatrixemptycode%
    \expandafter%
    \let% reinserted
  \else%
    \pgf@matrix@empty@cell@false%
    \expandafter%
    \pgfmatrixbegincode%
    \expandafter
    \let% reinserted
  \fi%
}   

\newif\ifpgf@matrix@empty@cell@


% End a  cell

\def\pgf@matrix@endcell{%
    \ifpgf@matrix@empty@cell@%
    \else%
      \expandafter\expandafter\expandafter\pgfmatrixendcode%
      \expandafter\ifpgf@matrix@last@cell@in@row\expandafter\pgf@matrix@last@cell@in@rowtrue\expandafter\fi%
    \fi%
    \pgfsys@endpicture%
    \ifnum\pgfmatrixcurrentcolumn<\pgf@matrix@numberofcolumns%
    \else%
    \ifpgf@matrix@last@cell@in@row\else%
    {%
      \global\pgf@matrix@fixedfalse%
      \pgf@x=0pt%
      \pgf@matrix@addtolength{\pgf@x}{\pgfmatrixcolumnsep}%
      \pgf@matrix@addtolength{\pgf@x}{\pgf@matrix@last@nextcell@options}%
      \ifpgf@matrix@fixed%
        \expandafter\pgfutil@g@addto@macro\csname pgf@matrix@column@finish@\the\pgfmatrixcurrentcolumn\endcsname%
          {\global\pgf@picmaxx=0pt}%
      \fi%
      \advance\pgfmatrixcurrentcolumn by1\relax % only temporary for the following:
      \expandafter\xdef\csname pgf@matrix@column@sep@\the\pgfmatrixcurrentcolumn\endcsname{\the\pgf@x}%
      \ifpgf@matrix@fixed%
        \expandafter\gdef\csname pgf@matrix@column@finish@\the\pgfmatrixcurrentcolumn\endcsname{\global\pgf@picminx=0pt}%
      \else%
        \expandafter\global\expandafter\let\csname pgf@matrix@column@finish@\the\pgfmatrixcurrentcolumn\endcsname=\pgfutil@empty%
      \fi%
    }%
    \fi\fi%  
    % Special case of the empty picture
    \ifdim\pgf@picmaxx=-16000pt\relax%
      \global\pgf@picmaxx=0pt\relax%
      \global\pgf@picminx=0pt\relax%
      \global\pgf@picmaxy=0pt\relax%
      \global\pgf@picminy=0pt\relax%
    \fi%
    \csname pgf@matrix@column@finish@\the\pgfmatrixcurrentcolumn\endcsname%
    \pgf@matrix@row@finish%
  \egroup\egroup%
  %
  % Step 5: Horizontal gap
  %
  \ifnum\pgfmatrixcurrentcolumn>1\relax%
    \pgf@y=\csname pgf@matrix@column@sep@\the\pgfmatrixcurrentcolumn\endcsname\relax%
    \advance\pgf@picminx by-\pgf@y%
  \fi%
  % 
  % Step 6: Protocol and adjust height and depth
  %
  % Step 6.1: Protocol...
  %
  \ifdim\pgf@picmaxy>\csname pgf@matrix@maxy\the\pgfmatrixcurrentrow\endcsname%
    \expandafter\xdef\csname pgf@matrix@maxy\the\pgfmatrixcurrentrow\endcsname{\the\pgf@picmaxy}%
  \fi%  
  \ifdim\pgf@picminy<\csname pgf@matrix@miny\the\pgfmatrixcurrentrow\endcsname%
    \expandafter\xdef\csname pgf@matrix@miny\the\pgfmatrixcurrentrow\endcsname{\the\pgf@picminy}%
  \fi%  
  %
  % Step 6.2: and setup.
  %  
  \ht\pgf@matrix@cell=\pgf@picmaxy%
  \dp\pgf@matrix@cell=-\pgf@picminy%
  %
  % Step 7: Protocol and adjust left and right width
  %
  % Step 7.1: Protocol...
  %
  \ifnum\pgfmatrixcurrentcolumn>\pgf@matrix@numberofcolumns\relax%
    \expandafter\xdef\csname pgf@matrix@maxx\the\pgfmatrixcurrentcolumn\endcsname{\the\pgf@picmaxx}%
    \expandafter\xdef\csname pgf@matrix@minx\the\pgfmatrixcurrentcolumn\endcsname{\the\pgf@picminx}%
    \global\pgf@matrix@numberofcolumns=\pgfmatrixcurrentcolumn\relax%
  \else%
    \ifdim\pgf@picmaxx>\csname pgf@matrix@maxx\the\pgfmatrixcurrentcolumn\endcsname%
      \expandafter\xdef\csname pgf@matrix@maxx\the\pgfmatrixcurrentcolumn\endcsname{\the\pgf@picmaxx}%
    \fi%  
    \ifdim\pgf@picminx<\csname pgf@matrix@minx\the\pgfmatrixcurrentcolumn\endcsname%
      \expandafter\xdef\csname pgf@matrix@minx\the\pgfmatrixcurrentcolumn\endcsname{\the\pgf@picminx}%
    \fi%  
  \fi%
  %
  % Step 7.2: and setup.
  %  
  \xdef\pgf@matrix@paddingskip{\the\pgf@picmaxx}%
  % 
  % Step 8: Put in the box now
  %
  \hfil\hskip-\pgf@picminx%
  \wd\pgf@matrix@cell=0pt%
  \box\pgf@matrix@cell%
  %
  % Step 9: End bounding box
  \endpgfinterruptboundingbox%
}

% Default code for the start and end code
\let\pgfmatrixbegincode=\relax
\let\pgfmatrixendcode=\relax
\let\pgfmatrixemptycode=\relax


% The following callback is called for every node that is produced
% inside a cell

\def\pgf@matrix@nodecallback#1{%
  \def\pgf@temp{#1}%
  \ifx\pgf@temp\pgfutil@empty%
  \else%
    \xdef\pgf@matrix@node@list{\pgf@matrix@node@list,{#1}}%
    \expandafter\xdef\csname pgf@matrix@node@location@#1\endcsname{{\the\pgfmatrixcurrentrow}{\the\pgfmatrixcurrentcolumn}}%
  \fi%
}



% Padding code

\def\pgf@matrix@padding{%
  \ifpgf@matrix@correct@call%
  \else%
    \pgferror{Single ampersand used with wrong catcode}%
  \fi%
  \hskip\pgf@matrix@paddingskip\hfil%
}


% Compute the real positions of the origins

% We must now compute the real positions of the origins of all the
% small pictures. To this end, we need to compute prefix sums. After
% the procedure is done, the minx and the maxy will contain the origin
% positions. 

\def\pgf@matrix@compute@origin{%
  %
  % Inverse prefix sum on the vertical positions
  %
  {%
    \ifnum\pgfmatrixcurrentrow>0\relax%
      \expandafter\gdef\csname pgf@matrix@miny\the\pgfmatrixcurrentrow\endcsname{0pt}%
    \fi
    \c@pgf@counta=\pgfmatrixcurrentrow\relax%
    \loop%
    \ifnum\c@pgf@counta>1\relax%
      \pgf@y=\csname pgf@matrix@miny\the\c@pgf@counta\endcsname\relax%
      \advance\pgf@y by\csname pgf@matrix@maxy\the\c@pgf@counta\endcsname\relax
      \advance\c@pgf@counta by-1\relax%
      \pgf@ya=\csname pgf@matrix@miny\the\c@pgf@counta\endcsname\relax%
      \advance\pgf@y by-\pgf@ya\relax%
      \expandafter\xdef\csname pgf@matrix@miny\the\c@pgf@counta\endcsname{\the\pgf@y}%
    \repeat%
  }%
  %
  % Prefix sum on the horizontal positions
  %
  {%
    \ifnum\pgf@matrix@numberofcolumns>0\relax%
      \pgf@x=\csname pgf@matrix@minx1\endcsname\relax%
      \pgf@x=-\pgf@x%
      \expandafter\xdef\csname pgf@matrix@minx1\endcsname{\the\pgf@x}%
    \fi
    \c@pgf@counta=1\relax%
    \loop%
    \ifnum\c@pgf@counta<\pgf@matrix@numberofcolumns\relax%
      \pgf@x=\csname pgf@matrix@minx\the\c@pgf@counta\endcsname\relax%
      \advance\pgf@x by\csname pgf@matrix@maxx\the\c@pgf@counta\endcsname\relax%
      \advance\c@pgf@counta by1\relax%
      \pgf@xa=\csname pgf@matrix@minx\the\c@pgf@counta\endcsname\relax%
      \advance\pgf@x by-\pgf@xa\relax%
      \expandafter\xdef\csname pgf@matrix@minx\the\c@pgf@counta\endcsname{\the\pgf@x}%
    \repeat%
  }%
}


% Shift the nodes to their origins

% The following procedure shifts all nodes in
% \pgf@matrix@node@list to their location inside a temporary
% picture. This picture will later be shifted again to its final
% position in the real picture.

\def\pgf@matrix@shift@nodes@initial{%
  {%
  \pgfutil@for\pgf@matrix@node@name:=\pgf@matrix@node@list\do{%
    \ifx\pgf@matrix@node@name\pgfutil@empty%
    \else%
      \expandafter\ifx\csname pgf@matrix@node@visited@\pgf@matrix@node@name\endcsname\relax%
        \pgf@shift@node{\pgf@matrix@node@name}{%
          \pgf@x=\csname pgf@matrix@minx%
            \expandafter\expandafter\expandafter\pgfutil@secondoftwo\csname pgf@matrix@node@location@\pgf@matrix@node@name\endcsname\endcsname%
          \pgf@y=\csname pgf@matrix@miny%
            \expandafter\expandafter\expandafter\pgfutil@firstoftwo\csname pgf@matrix@node@location@\pgf@matrix@node@name\endcsname\endcsname%
          }%
        \expandafter\let\csname pgf@matrix@node@visited@\pgf@matrix@node@name\endcsname=\pgfutil@empty%
      \fi%
    \fi%
  }%
  }%
}


% The second shifting, done in the following procedure, shifts all
% nodes to their real positions inside the real picture.

\def\pgf@matrix@shift@nodes@secondary#1{%
  \pgfutil@for\pgf@matrix@node@name:=\pgf@matrix@node@list\do{%
    \ifx\pgf@matrix@node@name\pgfutil@empty%
    \else%
      \expandafter\ifx\csname pgf@matrix@node@visited@\pgf@matrix@node@name\endcsname\relax%
        \pgf@shift@node{\pgf@matrix@node@name}{#1}%
        \expandafter\let\csname pgf@matrix@node@visited@\pgf@matrix@node@name\endcsname=\pgfutil@empty%
      \fi%
    \fi%
  }%
}


% End of line
\def\pgfmatrixendrow{%
  % if the cell contains nothing, the following \let will be at the
  % beginning (macro expansion has stopped here since neither \omit
  % nor \span was found)
  \let\pgf@matrix@signal@cell@end=\pgf@matrix@signal@cell@end
  \pgf@matrix@last@cell@in@rowtrue%
  &\pgf@matrix@correct@calltrue%
  \global\pgf@matrix@fixedfalse%
  \pgf@y=0pt%
  \pgf@matrix@addtolength\pgf@y{\pgfmatrixrowsep}%
  \pgfutil@ifnextchar[{\pgfmatrixendrow@skip}{\pgf@matrix@finish@line}%
}
\newif\ifpgf@matrix@last@cell@in@row

\def\pgf@matrix@signal@cell@end{\pgf@matrix@signal@cell@end}
% exact definition does not matter, only needs to be unique.

\def\pgfmatrixendrow@skip[#1]{%
  \pgf@matrix@addtolength\pgf@y{#1}%
  \pgf@matrix@finish@line%
}

\def\pgf@matrix@finish@line{%
  \global\pgf@y=\pgf@y%
  \pgf@ya=-\pgf@y%
  \global\advance\pgf@ya by\csname pgf@matrix@miny\the\pgfmatrixcurrentrow\endcsname%
  \expandafter\xdef\csname pgf@matrix@miny\the\pgfmatrixcurrentrow\endcsname{\the\pgf@ya}%
  \pgfutil@ifnextchar\egroup{\cr\pgf@end@matrix}{\pgf@matrix@no@eom@found}%
}

\def\pgf@matrix@eom@found{%
  \cr%
  \pgf@end@matrix%
}

\def\pgf@matrix@no@eom@found{%
  \cr%
  \noalign{\vskip\pgf@y%
  \ifpgf@matrix@fixed%
    \vskip\csname pgf@matrix@miny\the\pgfmatrixcurrentrow\endcsname% unskip
    \vskip\pgf@y%
    \pgf@y=-\pgf@y%
    \expandafter\xdef\csname pgf@matrix@miny\the\pgfmatrixcurrentrow\endcsname{\the\pgf@y}%
    \gdef\pgf@matrix@row@finish{\global\pgf@picmaxy=0pt}%
  \else%
    \global\let\pgf@matrix@row@finish=\pgfutil@empty%
  \fi%
  }%
}


\endinput
